%!TEX program = xelatex

\documentclass{uniquecv}

\usepackage{fontawesome5}
\usepackage{fontspec}
\usepackage[colorlinks]{hyperref}
% ----------------------------------------------------------------------------- %

%\setmonofont{FiraCode}
\begin{document}
\name{XXX}
\medskip

\basicinfo{
  \faPhone ~ (+86) 158-7896-1547
  \textperiodcentered\
  \faEnvelope ~ 2@qq.com
  \textperiodcentered\
  \faGithub ~ github.com/cheerryio
}

% ----------------------------------------------------------------------------- %

\section{教育背景}
\dateditem{\textbf{重庆} \quad 初中、高中}{2012年 -- 2018年 }
\dateditem{\textbf{aaaa} \quad aaa \quad 本科}{2018年 -- 至今 }
\quad 英语:CET4 \quad CET6


% ----------------------------------------------------------------------------- %

\section{专业技能}
\smallskip
\textbf{Programming}
\quad C/C++/Python/Javascript/Assembly x86/Java/TypeScript/React/Node.js/PHP

Linux/操作系统/数据结构/算法设计/计算机网络/

\textbf{Tools} \quad \LaTeX/Git

% ----------------------------------------------------------------------------- %

\section{个人简介}

能够熟练使用Linux,有扎实的计算机理论基础,喜欢钻研底层原理。

% ----------------------------------------------------------------------------- %

\section{课内实验}

\datedproject{汇编语言商店}{masm32}{2020}
{\it 一个简易电商系统}
项目地址\quad \href{https://github.com/cheerryio/HHHHHHust-awful3/tree/assemblystore}{{\color{gray}{\faLink}}~}
\begin{itemize}
  \item DOS汇编的编写与调试
  \item masm32汇编的编写与调试
\end{itemize}

\datedproject{计算机网络}{课内项目}{2020}
{\it TCP/IP模型各层部分算法的理解,实现,应用}
项目地址 \quad \href{https://github.com/cheerryio/HHHHHHust-awful3/tree/network}{{\color{gray}{\faLink}}~}
\vspace{0.4ex}
\begin{itemize}
  \item 基于winsock2实现静态http服务器
  \item 实现运输层选择重传SR协议与GBN协议
  \item 使用packetTracer软件模拟小型校园网络的搭建
\end{itemize}

\datedproject{大数据处理}{课程实验}{2020}
\textit{搭建hadoop的docker集群}
项目地址\quad \href{https://github.com/cheerryio/HHHHHHust-awful3/tree/dataprocess}{{\color{gray}{\faLink}}~}
\begin{itemize}
  \item Dockerfile的编写与使用
  \item hadoop配置文件的编写
\end{itemize}

\datedproject{大数据分析}{课程实验}{2020}
{\it 几种大数据分析算法的实现}
项目地址 \quad \href{https://github.com/cheerryio/HHHHHHust-awful3/tree/datamining}{{\color{gray}{\faLink}}~}
\begin{itemize}
  \item 利用多线程模拟模拟mapReduce多个节点的实现
  \item pageRank算法实现
  \item apriori算法实现关联关系挖掘
  \item kmeans聚类算法实现
  \item 基于用户的协同过滤和基于内容的推荐系统
\end{itemize}

\datedproject{数独小游戏}{课程实验}{2020}
{\it web端的小游戏}
项目地址 \quad \href{https://github.com/cheerryio/HHHHHHust-awful3/tree/software}{{\color{gray}{\faLink}}~}
\quad \emph{核心开发人员}
\begin{itemize}
  \item 技术栈为 React/TypeScript/Redux/RxJS/Material UI
  \item 负责项目目录结构的规划设计
  \item 负责核心组件的逻辑编写
\end{itemize}

\end{document}
